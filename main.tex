\documentclass{article}
\usepackage[utf8]{inputenc}
\usepackage[norsk]{babel}
\usepackage{listings}
\usepackage{verbatim}
\usepackage{color}
\usepackage{newclude}
\usepackage{enumerate}
\usepackage{hyperref}

\definecolor{codegreen}{rgb}{0,0.6,0}
\definecolor{codegray}{rgb}{0.5,0.5,0.5}
\definecolor{codepurple}{rgb}{0.58,0,0.82}
\definecolor{backcolour}{rgb}{0.95,0.95,0.92}
\newcommand\horizontalrule{\noindent\makebox[\linewidth]{\color{backcolour}\rule{\textwidth}{0.4pt}}} 
 
\lstdefinestyle{mystyle}{
    backgroundcolor=\color{backcolour},   
    commentstyle=\color{codegreen},
    keywordstyle=\color{magenta},
    numberstyle=\tiny\color{codegray},
    stringstyle=\color{codepurple},
    basicstyle=\footnotesize,
    breakatwhitespace=false,         
    breaklines=true,                 
    captionpos=b,                    
    keepspaces=true,                 
    numbers=left,                    
    numbersep=5pt,                  
    showspaces=false,                
    showstringspaces=false,
    showtabs=false,                  
    tabsize=2
}
 
\lstset{style=mystyle}
\author{Simen Lybekk}
\title{LaTeX håndbok}
\date{February 2016}

\begin{document}

\maketitle

\tableofcontents
\clearpage


\section{Grunnleggende \LaTeX}
Denne seksjonen vil ta for seg oppbyggingen av \LaTeX-syntaksen, de forskjellige mekanismene som eksisterer og hvordan de brukes.
Formålet her blir å introdusere generelle konsepter som \LaTeX~har, som for eksempel ``environments''.


\section{Vanlig bruk av \LaTeX}
Denne seksjonen vil ta for seg meget enkle beskrivelser om hvordan man kan foreta generell formatering av tekst og informasjon med \LaTeX.


\subsection{Nummererte lister (\texttt{enumerate}-miljøet)}
\label{env:enumerate}
\begin{description}
    \item[Pakke] Globalt - Pakken \texttt{enumerate} kan benyttes for å endre nummereringsmetoden.
    \item[Mer info] \texttt{texdoc enumerate} - For info om bruk av den valgfrie pakken og alternativ nummerering.
\end{description}

Nummererte lister blir lagd ved å benytte seg av \texttt{enumerate}-miljøet.
Hvert \texttt{\textbackslash item} i dette miljøet vil dukke opp som ett punkt i en nummerert liste.
\texttt{enumerate}-miljøet kan enkelt kombineres med \texttt{itemize}-miljøet og \texttt{enumerate}-miljøet for å lage lister med flere nivåer.

\subsubsection*{Se også}
\begin{description}
    \item[\ref{pkg:enumerate}~\nameref{pkg:enumerate}] for å finne ut hvordan du kan endre nummereringsstilen.
    \item[\ref{pkg:enumitem}~\nameref{pkg:enumitem}] for å finne ut hvordan du fleksibelt kan tilpasse nummererte lister.
\end{description}

\subsubsection*{Bruk}
Sett inn en enkel liste med tre elementer:
\vspace{0.75em}
\begin{lstlisting}[language=Tex]
\begin{enumerate}
    \item Foo
    \item Bar
    \item Baz
\end{enumerate}
\end{lstlisting}
\noindent Resultat:
\begin{enumerate}
    \item Foo
    \item Bar
    \item Baz
\end{enumerate}
\vspace{0.75em}


\horizontalrule


\vspace{0.75em}
\noindent Sett inn en liste med underelementer:
\vspace{0.75em}
\begin{lstlisting}[language=Tex]
\begin{enumerate}
    \item Foo
    \begin{enumerate}
        \item Bar
        \item Bat
    \end{enumerate}
    \item Baz
\end{enumerate}
\end{lstlisting}
\vspace{0.75em}

\noindent Resultat:
\begin{enumerate}
    \item Foo
    \begin{enumerate}
        \item Bar
        \item Bat
    \end{enumerate}
    \item Baz
\end{enumerate}
\subsection{Beskrivelser (\texttt{description}-miljøet)}
\label{env:description}
\begin{description}
    \item[Pakke] Globalt
\end{description}

\subsubsection*{Bruk}
Sett inn en enkel liste med tre elementer:
\vspace{0.75em}
\begin{lstlisting}[language=Tex]
\begin{description}
    \item[Foo] bar
    \item[Baz] qux
\end{description}
\end{lstlisting}

\noindent Resultat:
\begin{description}
    \item[Foo] bar
    \item[Baz] qux
\end{description}
\vspace{0.75em}


\horizontalrule


\noindent Sett inn en beskrivelse med en liste:
\vspace{0.75em}
\begin{lstlisting}[language=Tex]
\begin{description}
    \item[Foo]
    \begin{enumerate}
        \item Baz
        \item Qux
    \end{enumerate}
    \item[Bar] Baz
\end{description}
\end{lstlisting}
\vspace{0.75em}

\noindent Resultat:
\begin{description}
    \item[Foo]
    \begin{enumerate}
        \item Baz
        \item Qux
    \end{enumerate}
    \item[Bar] Baz
\end{description}



\section{Andre ressurser}
Denne seksjonen vil ta for seg mer avansert typesetting i \LaTeX, og mer avansert bruk av \LaTeX-funksjoner og pakker.


\subsection{Alternativ nummerering av lister med \texttt{enumerate}-pakken}
\label{pkg:enumerate}
\begin{description}
    \item[Pakke] \texttt{enumerate}
    \item[Mer info] \texttt{texdoc enumerate}
    \item[Er i konflikt med] \texttt{enumitem}
\end{description}

Om det ønskes en annen nummereringsstil enn det malen din tilbyr kan du bruke pakken \texttt{enumerate}.
Denne pakken vil utvide \texttt{enumerate}-miljøet med ett valgfritt argument som beskriver etiketten som skal bli brukt.
Dette argumentet er en streng hvor plassholderne \texttt{I, i, A, a} og \texttt{1} vil erstattes med indeksen for det gjeldende
elementet.

\textbf{Merk!} Om det ønskes å bruke disse tegnene i etiketten uten at de blir erstattet må de plasseres innenfor \{\}.

\subsubsection*{Bruk}
Sett inn en enkel liste med tre elementer:
\vspace{0.75em}
\begin{lstlisting}[language=Tex]
\begin{enumerate}[i]
    \item Foo
    \item Bar
    \item Baz
\end{enumerate}
\end{lstlisting}

\noindent Resultat:
\begin{enumerate}[i]
    \item Foo
    \item Bar
    \item Baz
\end{enumerate}
\vspace{0.75em}


\horizontalrule


\vspace{0.75em}
\noindent Sett inn en liste med underelementer, hvor topplisten blir nummerert med romertall og underlisten med \texttt{alpha =}:
\vspace{0.75em}
\begin{lstlisting}[language=Tex]
\begin{enumerate}[I.]
    \item Foo
    \begin{enumerate}[a =]
        \item Bar
        \item Bat
    \end{enumerate}
    \item Baz
\end{enumerate}
\end{lstlisting}
\vspace{0.75em}

\noindent Resultat:
\begin{enumerate}[I.]
    \item Foo
    \begin{enumerate}[a =]
        \item Bar
        \item Bat
    \end{enumerate}
    \item Baz
\end{enumerate}

\subsection{Avansert tilpassing av nummererte lister med \texttt{enumitem}-pakken}
\label{pkg:enumitem}

TODO
\subsection{Inkludere PDFer}
\label{pkg:pdfpages}
\begin{description}
    \item[Pakke] \texttt{pdfpages}
    \item[Eksempel] \texttt{\textbackslash includepdf[pages=\{1-4\}]\{example.pdf\}}
    \item[Mer info] \texttt{texdoc pdfpages}
\end{description}

Å inkludere PDFer gjøres ved å benytte seg av \texttt{\textbackslash includepdf}-kommandoen.
Kommandoen har to argumenter som vi må forholde oss til: Et option \texttt{pages} som beskriver sidene som skal puttes inn i dokumentet, og filen som skal inkluderes.

\texttt{pages}-valget er en komma-separert liste med sider som skal inkluderes i dokumentet, i den rekkefølgen de oppstår i listen.
Man kan angi en rekke med sider ved å bruke \texttt{startside-sluttside} i listen, eller alle sider i PDFen ved å sette \texttt{pages} til \texttt{-}, altså en enkelt bindestrek.
Man kan også bruke \texttt{\{\}} for å sette inn en tom side.

\textbf{Merk!} \texttt{pages}-valget defaulter til en verdi av \texttt{1}, hvilket bare vil inkludere den første siden av PDFen.

\subsubsection*{Bruk}
Sett inn en hel PDF på det gjeldende stedet i dokumentet:
\vspace{0.75em}
\begin{lstlisting}
\includepdf[pages=-]{example.pdf}
\end{lstlisting}
\vspace{1.5em}

Sett inn side 1-3, side 5, en blank side og side 6 til siste side av PDFen:
\vspace{0.75em}
\begin{lstlisting}
\includepdf[pages={1-3,5,{},6-}]{example.pdf}
\end{lstlisting}
\vspace{1.5em}

\end{document}
