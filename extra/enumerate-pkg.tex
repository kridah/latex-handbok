\subsection{Alternativ nummerering av lister med \texttt{enumerate}-pakken}
\label{pkg:enumerate}
\begin{description}
    \item[Pakke] \texttt{enumerate}
    \item[Mer info] \texttt{texdoc enumerate}
    \item[Er i konflikt med] \texttt{enumitem}
\end{description}

Om det ønskes en annen nummereringsstil enn det malen din tilbyr kan du bruke pakken \texttt{enumerate}.
Denne pakken vil utvide \texttt{enumerate}-miljøet med ett valgfritt argument som beskriver etiketten som skal bli brukt.
Dette argumentet er en streng hvor plassholderne \texttt{I, i, A, a} og \texttt{1} vil erstattes med indeksen for det gjeldende
elementet.

\textbf{Merk!} Om det ønskes å bruke disse tegnene i etiketten uten at de blir erstattet må de plasseres innenfor \{\}.

\subsubsection*{Bruk}
Sett inn en enkel liste med tre elementer:
\vspace{0.75em}
\begin{lstlisting}[language=Tex]
\begin{enumerate}[i]
    \item Foo
    \item Bar
    \item Baz
\end{enumerate}
\end{lstlisting}

\noindent Resultat:
\begin{enumerate}[i]
    \item Foo
    \item Bar
    \item Baz
\end{enumerate}
\vspace{0.75em}


\horizontalrule


\vspace{0.75em}
\noindent Sett inn en liste med underelementer, hvor topplisten blir nummerert med romertall og underlisten med \texttt{alpha =}:
\vspace{0.75em}
\begin{lstlisting}[language=Tex]
\begin{enumerate}[I.]
    \item Foo
    \begin{enumerate}[a =]
        \item Bar
        \item Bat
    \end{enumerate}
    \item Baz
\end{enumerate}
\end{lstlisting}
\vspace{0.75em}

\noindent Resultat:
\begin{enumerate}[I.]
    \item Foo
    \begin{enumerate}[a =]
        \item Bar
        \item Bat
    \end{enumerate}
    \item Baz
\end{enumerate}
