\subsection{Inkludere PDFer}
\label{pkg:pdfpages}
\begin{description}
    \item[Pakke] \texttt{pdfpages}
    \item[Eksempel] \texttt{\textbackslash includepdf[pages=\{1-4\}]\{example.pdf\}}
    \item[Mer info] \texttt{texdoc pdfpages}
\end{description}

Å inkludere PDFer gjøres ved å benytte seg av \texttt{\textbackslash includepdf}-kommandoen.
Kommandoen har to argumenter som vi må forholde oss til: Et option \texttt{pages} som beskriver sidene som skal puttes inn i dokumentet, og filen som skal inkluderes.

\texttt{pages}-valget er en komma-separert liste med sider som skal inkluderes i dokumentet, i den rekkefølgen de oppstår i listen.
Man kan angi en rekke med sider ved å bruke \texttt{startside-sluttside} i listen, eller alle sider i PDFen ved å sette \texttt{pages} til \texttt{-}, altså en enkelt bindestrek.
Man kan også bruke \texttt{\{\}} for å sette inn en tom side.

\textbf{Merk!} \texttt{pages}-valget defaulter til en verdi av \texttt{1}, hvilket bare vil inkludere den første siden av PDFen.

\subsubsection*{Bruk}
Sett inn en hel PDF på det gjeldende stedet i dokumentet:
\vspace{0.75em}
\begin{lstlisting}
\includepdf[pages=-]{example.pdf}
\end{lstlisting}
\vspace{1.5em}

Sett inn side 1-3, side 5, en blank side og side 6 til siste side av PDFen:
\vspace{0.75em}
\begin{lstlisting}
\includepdf[pages={1-3,5,{},6-}]{example.pdf}
\end{lstlisting}
\vspace{1.5em}